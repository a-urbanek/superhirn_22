%start preamble
\documentclass[paper=a4,fontsize=11pt,DIV14]{scrartcl}%kind of doc, font size, paper size
\usepackage[ngerman]{babel}%for special german letters etc			
%\usepackage{t1enc} obsolete, but some day we go back in time and could use this again
\usepackage[T1]{fontenc}%same as t1enc but better						
\usepackage[utf8]{inputenc}%utf-8 encoding, other systems could use others encoding		
\usepackage{amsmath}%get math done
\usepackage{graphicx}%get pictures & graphics done
\graphicspath{{pictures/}}%folder to stash all kind of pictures etc
\usepackage[pdftex,hidelinks]{hyperref}%for links to web
\usepackage{amssymb}%symbolics for math
\usepackage{amsfonts}%extra fonts
\usepackage []{natbib}%citation style
\usepackage{caption}%captions under everything
\usepackage{listings}
\usepackage[titletoc]{appendix}
\numberwithin{equation}{section} 
\usepackage[printonlyused,withpage]{acronym}%how to handle acronyms
\usepackage{float}%for garphics and how to let them floating around in the doc
\usepackage{cclicenses}%license!
\usepackage{xcolor}%nicer colors, here used for links
\usepackage{wrapfig}%making graphics floated by text and not done by minipage
\pdfpkresolution=2400%higher resolution
\usepackage{lipsum}%dummy text -- delete if dummies removed

%settings colors for links
\hypersetup{
    colorlinks,
    linkcolor={blue!50!black},
    citecolor={blue},
    urlcolor={blue!80!black}
}

%\usepackage[pagetracker=true]{biblatex}

%%starts with title page%%
\begin{document}
\bibliographystyle{alpha}

\begin{titlepage}
%deckblatt start
\thispagestyle{empty}
\begin{center}
\includegraphics[width=0.45\textwidth]{HTW_Logo_rgb}\\
%\Large{Hochschule für Technik und Wirtschaft Berlin (HTW-Berlin)}\\
\end{center}
 
 
\begin{center}
\Large{Fachbereich 4}\\
\textbf{Studiengang Angewandte Informatik, BSc}
\end{center}
\begin{verbatim}
 
 
\end{verbatim}
\begin{center}
\textbf{\LARGE{Pflichtenehft}}\\
\end{center}
\begin{verbatim}

\end{verbatim}
\begin{center}
\textbf{\Large{Sommersemester 2023}}\\
\end{center}

 
\begin{flushleft}
\begin{tabular}{lll}
& & \\
\textbf{Thema:} & & Softwareprojekt "Mastermind"\\
& & \\
& & \\
\textbf{Ansprechpartner:} & & Prof. Dr. Salinger\\
& & \\
& & \\
& & \\
\textbf{eingereicht von:} 	& & Adrian Urbanek, 563284\\
& & \\
& & \\
& & \\
& & \\
%%\textbf{eingereicht am:} & & \today \\
\text{Berlin, den \today}\\
& & \\
& & \\
& & \\
& & \\
& & \\
\hspace*{\fill}\begin{tabular}{@{}l@{}}\hline
\makebox[6cm]{Praktikant/in}
\end{tabular}
&   &
\hspace*{\fill}\begin{tabular}{@{}l@{}}\hline
\makebox[6cm]{Betreuer/in}
\end{tabular}
\end{tabular}
\end{flushleft}
\end{titlepage}

\newpage
\tableofcontents
\newpage

\section{Zielbestimmung}
\begin{itemize}
	\item Das Ziel dieses Projekts ist die Softwareentwicklung eines Mastermind-Spiels mit grafischer Oberfläche auf einer Python3-Laufzeitumgebung
\end{itemize}
\subsection{Musskriterien}
\begin{itemize}
	\item Objektorientierte Implementierung der Software in der Programmierumgebung \textit{python3} 
	\item Applikation mit grafischer Oberfläche (GUI) 
	\item Auswahl der Spielmodi \textit{Codemaker} und \textit{Codebreaker}
	\item Einfache Handhabung der Applikation 
	\item Ausführbarkeit der Applikation auf allen gängigen Betriebssystemen (Mac, Windows, Linux)
\end{itemize}
\subsection{Wunschkriterien}
\begin{itemize}
	\item Grafische Oberfläche im Retro-Design
	\item Zeitangabe, wie lange ein Spiel gedauert hat
\end{itemize}
\subsection{Abgrenzungskriterien}
\begin{itemize}
	\item Spielmodus ausschließlich Human vs. Computer
	\item Software soll keine Webapplikation sein
	\item Nur ein festgelegter Schwierigkeitsgrad
	\item Plattformunterstützung nur für Heimcomputer und Laptops
	\item Die Speicherung eines Spielstands ist nicht nötig
\end{itemize}
\section{Produkteinsatz}
\subsection{Anwendungsbereiche}
\begin{itemize}
    \item Unterhaltung und Freizeit
    \item Das Produkt ist für den privaten Gebrauch am Heimcomputer/Laptop vorgesehen.
\end{itemize}
\subsection{Zielgruppen}
\begin{itemize}
    \item Alle Personen der Altersgruppe 6 bis 40 mit Affinität für Computerspiele und Spaß an logischen Rätseln.
\end{itemize}

\subsection{Betriebsbedingungen}
\begin{itemize}
    \item Funktionsfähige Python3-Laufzeitumgebung 
    \item Heimcomputer oder Laptop
\end{itemize}
\section{Produktübersicht}
Das Spiel soll durch Ausführung einer Installationsdatei auf einer entsprechenden
Betriebssystemumgebung mit den benötigten Paketen/Abhängigkeiten installiert werden.
Hierzu sollte eine Internetverbindung am Endgerät bestehen, um diese Abhängigkeiten
herunterladen zu können. Es werden keine Administratorrechte für die Installation nötig sein.
Sobald das Spiel erfolgreich installiert wurde, kann es vom Nutzer ausgeführt werden.
(siehe Use-Case-Diagramm)
\subsection{Spielablauf}
\begin{itemize}
    \item Der Codemaker wählt eine geheime Kombination aus Farben und Positionen.
    \item Der Codebreaker versucht genau diese Kombination und Farbfolge zu erraten.
    \item Der Codemaker gibt Feedback über die Übereinstimmung des Rateversuchs über ein weißes Stäbchen (richtige Kugel an falscher Stelle) oder ein schwarzes (richtige Kugel an richtiger Stelle).
    \item Welche Kugeln richtig liegen, wird dem Codebreaker nicht verraten.
    \item Dieser Prozess wird wiederholt, bis entweder die geheime Kombination erraten oder die maximale Anzahl von Zügen erreicht wurde.
\end{itemize}
\section{Produktfunktionen}
\begin{itemize}
    \item Installation/Deinstallation der Anwendung durch den Benutzer
    \item Starten/Beenden der Anwendung durch den Benutzer
    \item Starten/Beenden einer Spielrunde
    \item Auswahl des Spielmodus Codemaker/Codebreaker im Hauptmenü der Anwendung
    \item Hilfestellung zum Spiel durch Tooltips in der Anwendung
    \item Zeitangabe zur Dauer eines Spiels
\end{itemize}
\section{Produktdaten}
(Siehe Objektdiagramm)
\section{Produktleistungen}
???
\section{Qualitätsanforderungen}
\begin{itemize}
    \item Das Programm soll keine Prozessabstürze verursachen.
    \item Bedienfehler sollen auf intuitive Weise abgefangen werden.
    \item Dokumentation soll erstellt werden.
\end{itemize}
\section{Benutzeroberfläche}
\begin{itemize}
    \item Das Spiel soll ein Hauptmenü mit Spielstart, Beenden der Anwendung und Auswahl des Spielmodus beinhalten.
    \item Die Benutzeroberfläche soll benutzerfreundlich und intuitiv sein.
    \item Die Darstellung der zu erratenen Kugeln wird durch farbige Kreise erfolgen.
\end{itemize}
\section{Nichtfunktioniale Anforderungen}
\begin{itemize}
    \item Plattformunabhängigkeit: Das Spiel soll auf verschiedenen Betriebssystemen (Windows, macOS, Linux) lauffähig sein.
    \item Benutzerfreundlichkeit: Die Benutzeroberfläche soll intuitiv und einfach zu bedienen sein, auch für unerfahrene Benutzer.
    \item Sicherheit: Das Spiel soll keine Sicherheitsrisiken aufweisen
    \item Wartbarkeit: Der Code soll gut strukturiert, dokumentiert und wartungsfreundlich sein, um zukünftige Änderungen und Erweiterungen zu erleichtern.
\end{itemize}
\section{Technische Produktumgebung}
\begin{itemize}
    \item Progammiersprache: Python 3
    \item Verwendung des GUI-Frameworks "Pygame"
    \item Ggf. Ausführung in virtuellen Python-Umgebung (venv)
\end{itemize}
\section{Spezielle Anforderungen and die Entwicklungsumgebung}
\begin{itemize}
    \item Visuelle Darstellung des Projekt-Workflows durch das Tool "Trello"
    \item Versionierung und gemeinsame Nutzung des Quellcodes über geeignete Tools (z. B. Git, Htw-GitLab)
    \item Pairprogramming für die Implementierung der Software
\end{itemize}
\section{Ergänzungen}




\end{document}